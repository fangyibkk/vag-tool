\documentclass{article}
\usepackage{geometry}
\geometry{
  margin=4cm
}
\author{Tanut Aranchayanont}
\usepackage{dirtree}

\title{VAG tool documentation}
\begin{document}
\maketitle

This is an update summary of VAG signal processing scripts used in the article \emph{Spectral Analysis on Vibroartrographic Signal of Total Knee Arthroplasty}. There are two issues intended to solve in this version:
\begin{itemize}
  \item Break the file allocation dependency. Former script use file allocation convention to run script properly, \emph{i.e.}, the new data set need to be place in the correct directory, for example, \texttt{../../data/}.
  \item Use R library for plotting. \texttt{ggplot} library provides a publication-quality plotting library and is open source (in progress).
\end{itemize}
The overview of all code and folder structure is shown below:
\dirtree{%
.1 vag-tool.
.2 doc // documentation latex source code.
.2 lib // external library used to write up the script.
.3 cdfa.
.3 gsl-2.2.1.
.3 libeemd.
.3 README.
.2 src // code written in this project.
.3 signal\_to\_imfs.
.3 signal\_to\_imfs\_all.
.3 remove\_spike.
.3 script\_dfa.sh.
.3 use\_dfa.
.3 use\_dfa\_all.
.3 $\ldots$.
}

\section*{Installation}
Prerequisite is scientific python and r which can be easily installed on Debian linux with the command:
\begin{verbatim}
$ sudo apt-get install build-essential python-pip3 python3 ipython3 r-base
$ pip install numpy scipy matplotlib
\end{verbatim}
Then copy the folder \texttt{vag-tool} to your computer. In order to make command line know the new script, we have to export the new path to the folder, which can be done easily by:
\begin{verbatim}
$ export PATH=$PATH:/path/to/vag-tool/src/
\end{verbatim}
Additionally, we can add this line to the \texttt{.bash\_profile} or \texttt{.bashrc} to make this line execute every time we attend the terminal session.

\section*{Usage}
\subsection*{EEMD with libeemd}
For IMFs, EEMD algorithm take a single time signal located at path, \texttt{in\_path} and output the set of IMF to the output directory \texttt{out\_dir}. In case multiple input files, there is a script called \texttt{signal\_to\_imfs\_all} which replace path to input file, \texttt{in\_path} with path to input directory \texttt{in\_dir}
\begin{verbatim}
$ signal_to_imfs in_path out_dir\\
$ signal_to_imfs_all in_dir out_dir
\end{verbatim}
\subsection*{DFA}
After downloading library DFA and \texttt{make} process, there will be an executable \texttt{dfa}. This exececutable can be used directly one by one to create DFA pairs but not very convenient when we would like to process over the directory so that \texttt{script\_dfa.sh} is introduced.\\
\begin{verbatim}
$ script\_dfa in_dir dfa_pair_out_dir
\end{verbatim}

The parameter $\alpha$ is calculated from DFA pairs. Each IMFs is composed back with the alpha criterion, for example, $\alpha > 0.5$. Both steps (1) obtain $\alpha$ of each IMF (2) composite the IMFs back due to alpha criterion are done with \texttt{use\_dfa} and \texttt{use\_dfa\_all}.
\begin{verbatim}
$ use_dfa_all imfs_dir dfa_pair_dir alpha_out_dir final_signal_dir
\end{verbatim}

\section*{Working example}
Let assume there are the data at directory \texttt{sample}:

\dirtree{%
.1 sample.
.2 s1.txt.
.2 s2.txt.
}
then processed with \texttt{signal\_to\_imfs\_all}
\begin{verbatim}
$ signal_to_imfs_all sample/ imfs/
\end{verbatim}
and finally out put to the folder \texttt{imfs}

\dirtree{%
.1 imfs.
.2 s1.
.3 0.txt.
.3 1.txt.
.3 2.txt.
.3 3.txt.
.3 4.txt.
.2 s2.
.3 0.txt.
.3 1.txt.
.3 2.txt.
.3 3.txt.
.3 4.txt.
}



\end{document}